% !TEX root = ../../ctfp-print.tex

\lettrine[lhang=0.17]{타}{입과 순수 함수를 카테고리로} 모델링하는 방법을 살펴 보았습니다.
또한 카테고리 이론에서 부작용이나 순수하지 않은 함수를 모델링하는 방법이 있다고 언급했습니다.
실행을 기록하거나 추적하는 함수와 같은 하나의 예를 살펴 보겠습니다. 명령형 언어에서는 다음과
같이 전역 상태를 변경하여 구현할 수 있습니다:

\begin{snip}{cpp}
string logger;

bool negate(bool b) {
    logger += "Not so! ";
    return !b;
}
\end{snip}
이를 메모하여 사용하면 로그를 생성하지 못하기 때문에 순수한 함수가 아니라는걸 알 수 있습니다.
이 함수는 \newterm{부작용}이 있습니다.

최근의 프로그래밍에서는 복잡한 동시성을 다루기에 가능한 한 전역 상태를 변경하는 것을 기피합니다.
아무도 이런 코드를 라이브러리에 결코 넣고 싶지 않겠죠.

다행히 우리는 이 함수를 순수하게 만들 수 있습니다. 로그를 명시 적으로 안팎으로 전달하면 됩니다.
문자열 인수를 추가하고 원래의 결과를 업데이트 된 로그를 포함한 문자열과 함께 출력하는걸 수행
하십시오:

\begin{snip}{cpp}
pair<bool, string> negate(bool b, string logger) {
    return make_pair(!b, logger + "Not so! ");
}
\end{snip}
이 함수는 순수하고 부작용이 없으며 동일한 인수의 호출에 동일한 결과를 반환하며 필요하다면 메모
할 수 있습니다. 그러나 로그를 누적하기 때문에 앞서 호출된 모든 가능한 기록을 메모해야 합니다.
다음과 같은 별도의 메모 항목이 있습니다:

\begin{snip}{cpp}
negate(true, "It was the best of times. ");
\end{snip}
그리고

\begin{snip}{cpp}
negate(true, "It was the worst of times. ");
\end{snip}
이런 식이죠.

또한 라이브러리로 쓰이기엔 훌륭한 인터페이스가 아닙니다. 호출하는 쪽에서 원하면 반환된 문자열을
무시할 수 있으므로 큰 부담이 되지 않습니다. 그러나 문자열을 입력으로 전달해야 하므로 불편할 수
있습니다.

이와같은 일을 좀 더 수월하게 해결 할 수 없을까요? 관심사를 분리해 보는 건 어떨까요? 이 간단한
예제에서 함수 negate 의 주요 목적은 주어진 불리언 값을 다른 불리언 값으로 바꾸는 것입니다.
로깅은 보조입니다. 물론, 기록되는 메시지는 함수와 관련이 있지만 메시지를 하나의 연속 로그로
집계하는 작업은 별도의 문제입니다. 여전히 함수가 문자열을 생성하기를 원하지만 로그를 생성하지
않도록 합시다. 타협의 방법은 다음과 같습니다:

\begin{snip}{cpp}
pair<bool, string> negate(bool b) {
    return make_pair(!b, "Not so! ");
}
\end{snip}
아이디어는 함수 호출 \emph{사이에} 로그가 집계된다는 것입니다.

이 작업을 수행하는 방법을 알아보기 위해 좀 더 현실적인 예를 들어보겠습니다. 문자열에서 소문자를
대문자로 바꾸는 함수가 하나 있습니다:

\begin{snip}{cpp}
string toUpper(string s) {
    string result;
    int (*toupperp)(int) = &toupper; // toupper is overloaded
    transform(begin(s), end(s), back_inserter(result), toupperp);
    return result;
}
\end{snip}
그리고 문자열을 공백을 기준으로 나눠 문자열 벡터로 만드는 함수도 있습니다:

\begin{snip}{cpp}
vector<string> toWords(string s) {
    return words(s);
}
\end{snip}
실제 작업은 words 라는 보조 함수에서 수행합니다.

\begin{snip}{cpp}
vector<string> words(string s) {
    vector<string> result{""};
    for (auto i = begin(s); i != end(s); ++i)
    {
        if (isspace(*i))
            result.push_back(""); 
        else
            result.back() += *i;
    }
    return result;
}
\end{snip}
이제 문자열을 위의 \code{toUpper} 함수와 \code{toWords} 함수로 변경한 결과를
반환값으로 전송하려 합니다.

\begin{figure}[H]
\centering
\includegraphics[width=0.3\textwidth]{images/piggyback.jpg}
\end{figure}
\noindent
이 함수들의 반환 값을 "꾸밀" 수 있습니다. 일반적으로 임의의 값 \code{A}와 문자열의 쌍으로
캡슐화하는 템플릿 \code{Writer}를 정의하는 방식으로 수행합니다:
We will ``embellish'' the return values of these functions. Let's do it
in a generic way by defining a template \code{Writer} that
encapsulates a pair whose first component is a value of arbitrary type
\code{A} and the second component is a string:

\begin{snip}{cpp}
template<class A>
using Writer = pair<A, string>;
\end{snip}
여기 꾸며진 함수들이 있습니다:

\begin{snip}{cpp}
Writer<string> toUpper(string s) {
    string result;
    int (*toupperp)(int) = &toupper;
    transform(begin(s), end(s), back_inserter(result), toupperp);
    return make_pair(result, "toUpper "); 
}

Writer<vector<string>> toWords(string s) { 
    return make_pair(words(s), "toWords ");
}
\end{snip}
이 두 함수를 결합하여 문자열을 대문자로 만들고 단어로 분할하는 다른 꾸며진 함수를 만들어 봅시다.
그 방법은 다음과 같습니다:

\begin{snip}{cpp}
Writer<vector<string>> process(string s) {
    auto p1 = toUpper(s);
    auto p2 = toWords(p1.first);
    return make_pair(p2.first, p1.second + p2.second);
}
\end{snip}
We have accomplished our goal: The aggregation of the log is no longer
the concern of the individual functions. They produce their own
messages, which are then, externally, concatenated into a larger log.

Now imagine a whole program written in this style. It's a nightmare of
repetitive, error-prone code. But we are programmers. We know how to
deal with repetitive code: we abstract it! This is, however, not your
run of the mill abstraction --- we have to abstract \newterm{function
composition} itself. But composition is the essence of category theory,
so before we write more code, let's analyze the problem from the
categorical point of view.

\section{The Writer Category}

The idea of embellishing the return types of a bunch of functions in
order to piggyback some additional functionality turns out to be very
fruitful. We'll see many more examples of it. The starting point is our
regular category of types and functions. We'll leave the types as
objects, but redefine our morphisms to be the embellished functions.

For instance, suppose that we want to embellish the function
\code{isEven} that goes from \code{int} to \code{bool}. We turn it
into a morphism that is represented by an embellished function. The
important point is that this morphism is still considered an arrow
between the objects \code{int} and \code{bool}, even though the
embellished function returns a pair:

\begin{snip}{cpp}
pair<bool, string> isEven(int n) {
    return make_pair(n % 2 == 0, "isEven ");
}
\end{snip}
By the laws of a category, we should be able to compose this morphism
with another morphism that goes from the object \code{bool} to
whatever. In particular, we should be able to compose it with our
earlier \code{negate}:

\begin{snip}{cpp}
pair<bool, string> negate(bool b) {
    return make_pair(!b, "Not so! ");
}
\end{snip}
Obviously, we cannot compose these two morphisms the same way we compose
regular functions, because of the input/output mismatch. Their
composition should look more like this:

\begin{snip}{cpp}
pair<bool, string> isOdd(int n) {
    pair<bool, string> p1 = isEven(n);
    pair<bool, string> p2 = negate(p1.first);
    return make_pair(p2.first, p1.second + p2.second);
}
\end{snip}
So here's the recipe for the composition of two morphisms in this new
category we are constructing:

\begin{enumerate}
\tightlist
\item
  Execute the embellished function corresponding to the first morphism
\item
  Extract the first component of the result pair and pass it to the
  embellished function corresponding to the second morphism
\item
  Concatenate the second component (the string) of the first result
  and the second component (the string) of the second result
\item
  Return a new pair combining the first component of the final result
  with the concatenated string.
\end{enumerate}

If we want to abstract this composition as a higher order function in
C++, we have to use a template parameterized by three types
corresponding to three objects in our category. It should take two
embellished functions that are composable according to our rules, and
return a third embellished function:

\begin{snip}{cpp}
template<class A, class B, class C>
function<Writer<C>(A)> compose(function<Writer<B>(A)> m1,
                               function<Writer<C>(B)> m2)
{
    return [m1, m2](A x) {
        auto p1 = m1(x);
        auto p2 = m2(p1.first);
        return make_pair(p2.first, p1.second + p2.second); 
    };
}
\end{snip}
Now we can go back to our earlier example and implement the composition
of \code{toUpper} and \code{toWords} using this new template:

\begin{snip}{cpp}
Writer<vector<string>> process(string s) { 
    return compose<string, string, vector<string>>(toUpper, toWords)(s);
}
\end{snip}
There is still a lot of noise with the passing of types to the
\code{compose} template. This can be avoided as long as you have a
C++14-compliant compiler that supports generalized lambda functions with
return type deduction (credit for this code goes to Eric Niebler):

\begin{snip}{cpp}
auto const compose = [](auto m1, auto m2) { 
    return [m1, m2](auto x) { 
        auto p1 = m1(x);
        auto p2 = m2(p1.first);
        return make_pair(p2.first, p1.second + p2.second);
    };
};
\end{snip}
In this new definition, the implementation of \code{process}
simplifies to:

\begin{snip}{cpp}
Writer<vector<string>> process(string s) {
    return compose(toUpper, toWords)(s);
}
\end{snip}
But we are not finished yet. We have defined composition in our new
category, but what are the identity morphisms? These are not our regular
identity functions! They have to be morphisms from type A back to type
A, which means they are embellished functions of the form:

\begin{snip}{cpp}
Writer<A> identity(A);
\end{snip}
They have to behave like units with respect to composition. If you look
at our definition of composition, you'll see that an identity morphism
should pass its argument without change, and only contribute an empty
string to the log:

\begin{snip}{cpp}
template<class A> Writer<A> identity(A x) {
    return make_pair(x, "");
}
\end{snip}
You can easily convince yourself that the category we have just defined
is indeed a legitimate category. In particular, our composition is
trivially associative. If you follow what's happening with the first
component of each pair, it's just a regular function composition, which
is associative. The second components are being concatenated, and
concatenation is also associative.

An astute reader may notice that it would be easy to generalize this
construction to any monoid, not just the string monoid. We would use
\code{mappend} inside \code{compose} and \code{mempty} inside
\code{identity} (in place of \code{+} and \code{""}). There really
is no reason to limit ourselves to logging just strings. A good library
writer should be able to identify the bare minimum of constraints that
make the library work --- here the logging library's only requirement is
that the log have monoidal properties.

\section{Writer in Haskell}

The same thing in Haskell is a little more terse, and we also get a lot
more help from the compiler. Let's start by defining the \code{Writer}
type:

\src{snippet01}
Here I'm just defining a type alias, an equivalent of a \code{typedef}
(or \code{using}) in C++. The type \code{Writer} is parameterized by
a type variable \code{a} and is equivalent to a pair of \code{a} and
\code{String}. The syntax for pairs is minimal: just two items in
parentheses, separated by a comma.

Our morphisms are functions from an arbitrary type to some
\code{Writer} type:

\src{snippet02}
We'll declare the composition as a funny infix operator, sometimes
called the ``fish'':

\src{snippet03}
It's a function of two arguments, each being a function on its own, and
returning a function. The first argument is of the type
\code{(a -> Writer b)}, the second is
\code{(b -> Writer c)}, and the result is
\code{(a -> Writer c)}.

Here's the definition of this infix operator --- the two arguments
\code{m1} and \code{m2} appearing on either side of the fishy
symbol:

\src{snippet04}
The result is a lambda function of one argument \code{x}. The lambda
is written as a backslash --- think of it as the Greek letter λ with an
amputated leg.

The \code{let} expression lets you declare auxiliary variables. Here
the result of the call to \code{m1} is pattern matched to a pair of
variables \code{(y, s1)}; and the result of the call to \code{m2},
with the argument \code{y} from the first pattern, is matched to
\code{(z, s2)}.

It is common in Haskell to pattern match pairs, rather than use
accessors, as we did in C++. Other than that there is a pretty
straightforward correspondence between the two implementations.

The overall value of the \code{let} expression is specified in its
\code{in} clause: here it's a pair whose first component is \code{z}
and the second component is the concatenation of two strings,
\code{s1++s2}.

I will also define the identity morphism for our category, but for
reasons that will become clear much later, I will call it
\code{return}.

\src{snippet05}
For completeness, let's have the Haskell versions of the embellished
functions \code{upCase} and \code{toWords}:

\src{snippet06}
The function \code{map} corresponds to the C++ \code{transform}. It
applies the character function \code{toUpper} to the string
\code{s}. The auxiliary function \code{words} is defined in the
standard Prelude library.

Finally, the composition of the two functions is accomplished with the
help of the fish operator:

\src{snippet07}

\section{Kleisli Categories}

You might have guessed that I haven't invented this category on the
spot. It's an example of the so called Kleisli category --- a category
based on a monad. We are not ready to discuss monads yet, but I wanted
to give you a taste of what they can do. For our limited purposes, a
Kleisli category has, as objects, the types of the underlying
programming language. Morphisms from type $A$ to type $B$ are functions that
go from $A$ to a type derived from $B$ using the particular embellishment.
Each Kleisli category defines its own way of composing such morphisms,
as well as the identity morphisms with respect to that composition.
(Later we'll see that the imprecise term ``embellishment'' corresponds
to the notion of an endofunctor in a category.)

The particular monad that I used as the basis of the category in this
post is called the \newterm{writer monad} and it's used for logging or
tracing the execution of functions. It's also an example of a more
general mechanism for embedding effects in pure computations. You've
seen previously that we could model programming-language types and
functions in the category of sets (disregarding bottoms, as usual). Here
we have extended this model to a slightly different category, a category
where morphisms are represented by embellished functions, and their
composition does more than just pass the output of one function to the
input of another. We have one more degree of freedom to play with: the
composition itself. It turns out that this is exactly the degree of
freedom which makes it possible to give simple denotational semantics to
programs that in imperative languages are traditionally implemented
using side effects.

\section{Challenge}

A function that is not defined for all possible values of its argument
is called a partial function. It's not really a function in the
mathematical sense, so it doesn't fit the standard categorical mold. It
can, however, be represented by a function that returns an embellished
type \code{optional}:

\begin{snip}{cpp}
template<class A> class optional {
    bool _isValid;
    A _value;
public: 
    optional()    : _isValid(false) {}
    optional(A v) : _isValid(true), _value(v) {}
    bool isValid() const { return _isValid; }
    A value() const { return _value; }
};
\end{snip}
For example, here's the implementation of the embellished function
\code{safe\_root}:

\begin{snip}{cpp}
optional<double> safe_root(double x) {
    if (x >= 0) return optional<double>{sqrt(x)}; 
    else return optional<double>{};
}
\end{snip}
Here's the challenge:

\begin{enumerate}
\tightlist
\item
  Construct the Kleisli category for partial functions (define
  composition and identity).
\item
  Implement the embellished function \code{safe\_reciprocal} that
  returns a valid reciprocal of its argument, if it's different from
  zero.
\item
  Compose the functions \code{safe\_root} and \code{safe\_reciprocal} to implement
  \code{safe\_root\_reciprocal} that calculates \code{sqrt(1/x)}
  whenever possible.
\end{enumerate}
